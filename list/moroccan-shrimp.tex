\recipe{
    Moroccan Shrimp
}{
    \href{https://www.jamieoliver.com/recipes/seafood-recipes/sizzling-moroccan-prawns/}{Jamie Oliver}
}{
    \Ingredients{
        \item 2 sprigs of fresh rosemary
        \item 2 cloves of garlic
        \item olive oil
        \item 1 level teaspoon smoked paprika
        \item 1 good pinch of saffron
        \item 6 large raw shell-on king prawns, from sustainable sources
        \item 2 oranges
        \item 150 g wholewheat couscous
        \item 400 g colourful mixed seasonal veg, such as peas, asparagus, fennel, courgettes, celery, spring onions, red or yellow peppers
        \item 1 fresh red chilli
        \item 1/2 a bunch of fresh mint
        \item 1 lemon
        \item 2 tablespoons natural yoghurt
        \item 1 pomegranate
    }
}{
    \Directions{
        \item Strip the rosemary leaves into a pestle and mortar, then peel and add the garlic and pound into a paste with a pinch of sea salt.
        \item Muddle in 1 tablespoon of oil, the paprika, saffron and a swig of boiling water to make a marinade.
        \item Use little scissors to cut down the back of each prawn shell and remove the vein.
            Cut 1 orange into wedges, toss with the prawns and the marinade and leave aside for 10 minutes.
        \item Put the couscous into a bowl and just cover with boiling water, then pop a plate on top and leave to fluff up.
        \item Take a bit of pride in finely chopping all your colourful seasonal veg and chilli, and put them into a nice serving bowl.
        \item Pick a few pretty mint leaves and put to one side, then pick and finely chop the rest and add to the bowl with the juice of the lemon and the remaining orange.
            Add the couscous, toss together and season to perfection.
        \item Put a large non-stick frying pan on a high heat.
            Add the prawns, marinade and orange wedges and cook for 4 to 5 minutes, or until the prawns are gnarly and crisp, then arrange on top of the couscous.
        \item Dollop with yoghurt, then halve the pomegranate and, holding it cut side down in your fingers, bash the back so the sweet jewels tumble over everything.
            Sprinkle with the reserved mint leaves and serve.
    }
}
