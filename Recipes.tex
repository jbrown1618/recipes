\documentclass[oneside,11pt]{book}

\usepackage[top=1in, bottom=1in, left=1.25in, right=1.25in]{geometry}
\usepackage{parskip}			% Disables indentation at paragraph breaks
\usepackage{palatino}
\usepackage[]{units}

\newcommand{\Ingredients}[1]{
	\subsection*{Ingredients}
	\begin{itemize} 
	  #1 
	\end{itemize}
}

\newcommand{\Directions}[1]{
	\subsection*{Directions}
	\begin{enumerate} 
	  #1 
	\end{enumerate}
}

\newcommand{\recipe}[1]{
	\section*{ \hspace{-12pt} #1 }
	\addcontentsline{toc}{section}{ \hspace{-6pt} #1 }
}

\title{\bf \Huge Recipes}
\author{Joshua Brown}
\date{}

\begin{document}
\maketitle
\tableofcontents{}

\chapter{Indian}
	\recipe{Chicken Tikka Masala}
		\Ingredients{
			\item 2 tablespoons Ghee or butter
			\item 2 tablespoons ghee (clarified butter)
			\item 1 onion, finely chopped
			\item 4 cloves garlic, minced
			\item 1 tablespoon ground cumin
			\item 1 teaspoon salt
			\item 1 teaspoon ground ginger
			\item 1 teaspoon cayenne pepper
			\item 1/2 teaspoon ground cinnamon
			\item 1/4 teaspoon ground turmeric
			\item 1 (14 ounce) can tomato sauce
			\item 1 cup heavy whipping cream
			\item 2 teaspoons paprika
			\item 1 tablespoon white sugar
			\item 1 tablespoon vegetable oil
			\item 4 skinless, boneless chicken breast halves, cut into bite-size pieces
			\item 1/2 teaspoon curry powder
			\item 1/2 teaspoon salt, or to taste
			\item 1 teaspoon white sugar, or to taste
		}
		
		\Directions{
			\item Heat ghee in a large skillet over medium heat and cook and stir onion until translucent, about 5 minutes. 
			  Stir in garlic; cook and stir just until fragrant, about 1 minute. 
			  Stir cumin, 1 teaspoon salt, ginger, cayenne pepper, cinnamon, and turmeric into the onion mixture; 
			  fry until fragrant, about 2 minutes.
			\item Stir tomato sauce into the onion and spice mixture, bring to a boil, and reduce heat to low.
				Simmer sauce for 10 minutes, then mix in cream, paprika, and 1 tablespoon sugar. 
				Bring sauce back to a simmer and cook, stirring often, until sauce is thickened, 10 to 15 minutes.
			\item Heat vegetable oil in a separate skillet over medium heat. 
			  Stir chicken into the hot oil, sprinkle with curry powder, and sear chicken until lightly browned but still pink inside, about 3 minutes; stir often. 
			  Transfer chicken and any pan juices into the sauce. 
			  Simmer chicken in sauce until no longer pink, about 30 minutes; adjust salt and sugar to taste.
		}
		
	\recipe{Saag Paneer}
	  \Ingredients{
	    \item 1 teaspoon turmeric
      \item 1/2 teaspoon cayenne
      \item Kosher salt
      \item 3 tablespoons plus 1 1/2 tablespoons vegetable oil
      \item 12 ounces paneer, cut into 1-inch cubes
      \item 1 (16-ounce package) frozen chopped spinach
      \item 1 medium white onion, finely chopped
      \item 1 (1-inch thumb) ginger, peeled and minced (about 1 tablespoon)
      \item 4 cloves garlic, minced
      \item 1 large green serrano chile, finely chopped (seeds removed if you don't like it spicy!)
      \item 1/2 teaspoon store-bought or homemade garam masala, recipe follows
      \item 2 teaspoons ground coriander
      \item 1 teaspoon ground cumin
      \item 1/2 cup plain yogurt, stirred until smooth
	  }
	  
	  \Directions{
	    \item In a large bowl, whisk together the turmeric, cayenne, 1 teaspoon salt and 3 tablespoons oil. 
	      Gently, drop in the cubes of paneer and gently toss, taking care not to break the cubes if you're using the homemade kind. 
	      Let the cubes marinate while you get the rest of your ingredients together and prepped.
      \item Thaw the spinach in the microwave in a microwave-safe dish, 5 minutes on high, then puree in a food processor until smooth. 
        Alternatively, you can chop it up very finely with your knife.
      \item Place a large nonstick skillet over medium heat, and add the paneer as the pan warms. 
        In a couple of minutes give the pan a toss; each piece of paneer should be browned on one side. 
        Fry another minute or so, and then remove the paneer from the pan onto a plate.
      \item Add the remaining 1 1/2 tablespoons oil to the pan. 
        Add the onions, ginger, garlic and chile. 
        Now here's the important part: saute the mixture until it's evenly toffee-coloured, which should take about 15 minutes. 
        Don't skip this step - this is the foundation of the dish! 
        If you feel like the mixture is drying out and burning, add a couple of tablespoons of water.
      \item Add the garam masala, coriander and cumin. 
        If you haven't already, sprinkle a little water to keep the spices from burning. 
        Cook, stirring often, until the raw scent of the spices cook out, and it all smells a bit more melodious, 3 to 5 minutes.
      \item Add the spinach and stir well, incorporating the spiced onion mixture into the spinach. 
        Add a little salt and 1/2 cup of water, stir, and cook about 5 minutes with the lid off.
      \item Turn the heat off. 
        Add the yogurt, a little at a time to keep it from curdling. 
        Once the yogurt is well mixed into the spinach, add the paneer. 
        Turn the heat back on, cover and cook until everything is warmed through, about 5 minutes. Serve.
	  }
		
\chapter{Dessert}
  \recipe{Chocolate Lava Cake}
    \Ingredients{
      \item 6 (1-ounce) squares bittersweet chocolate
      \item 2 (1-ounce) squares semisweet chocolate
      \item 10 tablespoons (1 1/4 stick) butter
      \item 1/2 cup all-purpose flour
      \item 1 1/2 cups confectioners' sugar
      \item 3 large eggs
      \item 3 egg yolks
      \item 1 teaspoon vanilla extract
      \item 2 tablespoons orange liqueur
    }
    
    \Directions{
      \item Preheat oven to 425 degrees F. Grease 6 (6-ounce) custard cups. 
      \item Melt the chocolates and butter in the microwave, or in a double boiler. 
        Add the flour and sugar to chocolate mixture. Stir in the eggs and yolks until smooth. 
        Stir in the vanilla and orange liqueur. 
      \item Divide the batter evenly among the custard cups. 
        Place in the oven and bake for 14 minutes. 
        The edges should be firm but the center will be runny. 
        Run a knife around the edges to loosen and invert onto dessert plates.
    }
  
  \recipe{Tiramisu}
    \Ingredients{
      \item 6 egg yolks 
      \item 3/4 cup white sugar 
      \item 2/3 cup milk 
      \item 1 1/4 cups heavy cream 
      \item 1/2 teaspoon vanilla extract 
      \item 1 pound mascarpone cheese 
      \item 1/4 cup strong brewed coffee, room temperature
      \item 2 tablespoons rum 
      \item 2 (3 ounce) packages ladyfinger cookies 
      \item 1 tablespoon unsweetened cocoa powder
    }
    \Directions{
      \item In a medium saucepan, whisk together egg yolks and sugar until well blended. 
        Whisk in milk and cook over medium heat, stirring constantly, until mixture boils. 
        Boil gently for 1 minute, remove from heat and allow to cool slightly. 
        Cover tightly and chill in refrigerator 1 hour.
      \item In a medium bowl, beat cream with vanilla until stiff peaks form. 
        Whisk mascarpone into yolk mixture until smooth.
      \item In a small bowl, combine coffee and rum. 
        Split ladyfingers in half lengthwise and drizzle with coffee mixture.
      \item Arrange half of soaked ladyfingers in bottom of a 7x11 inch dish. 
        Spread half of mascarpone mixture over ladyfingers, then half of whipped cream over that. 
        Repeat layers and sprinkle with cocoa. Cover and refrigerate 4 to 6 hours, until set.
    }

\end{document}