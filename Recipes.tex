\documentclass[oneside,11pt]{book}

\usepackage[top=1in, bottom=1in, left=1.25in, right=1.25in]{geometry}
\usepackage{parskip}            % Disables indentation at paragraph breaks
\usepackage{palatino}
\usepackage[]{units}

\newcommand{\Ingredients}[1]{
    \subsection*{Ingredients}
    \begin{itemize} 
      #1 
    \end{itemize}
}

\newcommand{\Directions}[1]{
    \subsection*{Directions}
    \begin{enumerate} 
      #1 
    \end{enumerate}
}

\newcommand{\recipe}[1]{
    \section*{ \hspace{-12pt} #1 }
    \addcontentsline{toc}{section}{ \hspace{-6pt} #1 }
}

\title{\bf \Huge Recipes}
\author{Joshua Brown}
\date{}

\begin{document}
\maketitle
\tableofcontents{}

\chapter{Breakfast}
    \recipe{Crepes}
        \Ingredients{
            \item 2 large eggs
            \item 3/4 cup milk
            \item 1/2 cup water
            \item 1 cup flour
            \item 3 tablespoons melted butter
            \item Butter, for coating the pan
        }
        
        \Directions{
            \item In a blender, combine all of the ingredients and pulse for 10 seconds. 
                Place the crepe batter in the refrigerator for 1 hour. 
                This allows the bubbles to subside so the crepes will be less likely to tear during cooking. 
                The batter will keep for up to 48 hours.
            \item Heat a small non-stick pan. 
                Add butter to coat. 
                Pour 1 ounce of batter into the center of the pan and swirl to spread evenly. 
                Cook for 30 seconds and flip. Cook for another 10 seconds and remove to the cutting board. 
                Lay them out flat so they can cool. 
                Continue until all batter is gone. 
                After they have cooled you can stack them and store in sealable plastic bags in the refrigerator for several days or in the freezer for up to two months. 
                When using frozen crepes, thaw on a rack before gently peeling apart.
            \item Savory Variation: Add 1/4 teaspoon salt and 1/4 cup chopped fresh herbs, spinach or sun-dried tomatoes to the egg mixture.
            \item Sweet Variation: Add 21/2 tablespoons sugar, 1 teaspoon vanilla extract and 2 tablespoons of your favorite liqueur to the egg mixture.
        }
    
    \recipe{Buttermilk Pancakes}
        \Ingredients{
            \item 1 1/2 cups all purpose flour
            \item 3 tablespoons sugar
            \item 1 teaspoon baking powder
            \item 1/2 teaspoon baking soda
            \item 1 teaspoon salt
            \item 1 1/2 cups buttermilk
            \item 3 tablespoons butter, melted
            \item 2 eggs
            \item (1/2 teaspoon vanilla extract)
        }
        
        \Directions{
            \item Mix together the dry ingredients and wet ingredients in separate bowls.
            \item Grease a skillet lightly with butter or oil and do so as needed between batches. 
                The heat is correct when a few drops of cold water on the skillet bounce and sputter, not boiling or evaporating. 
            \item Mix the liquid quickly into the dry ingredients.
            \item Use 1/4 cup batter for each pancake. When bubbles appear on the surface on the pancake and the edges have browned, turn the cake and cook only until the second side is done.
        }

\chapter{Dinner}
    \recipe{Chicken Tikka Masala}
        \Ingredients{
            \item 2 tablespoons Ghee or butter
            \item 2 tablespoons ghee (clarified butter)
            \item 1 onion, finely chopped
            \item 4 cloves garlic, minced
            \item 1 tablespoon ground cumin
            \item 1 teaspoon salt
            \item 1 teaspoon ground ginger
            \item 1 teaspoon cayenne pepper
            \item 1/2 teaspoon ground cinnamon
            \item 1/4 teaspoon ground turmeric
            \item 1 (14 ounce) can tomato sauce
            \item 1 cup heavy whipping cream
            \item 2 teaspoons paprika
            \item 1 tablespoon white sugar
            \item 1 tablespoon vegetable oil
            \item 4 skinless, boneless chicken breast halves, cut into bite-size pieces
            \item 1/2 teaspoon curry powder
            \item 1/2 teaspoon salt, or to taste
            \item 1 teaspoon white sugar, or to taste
        }
        
        \Directions{
            \item Heat ghee in a large skillet over medium heat and cook and stir onion until translucent, about 5 minutes. 
                Stir in garlic; cook and stir just until fragrant, about 1 minute. 
                Stir cumin, 1 teaspoon salt, ginger, cayenne pepper, cinnamon, and turmeric into the onion mixture; 
                fry until fragrant, about 2 minutes.
            \item Stir tomato sauce into the onion and spice mixture, bring to a boil, and reduce heat to low.
                Simmer sauce for 10 minutes, then mix in cream, paprika, and 1 tablespoon sugar. 
                Bring sauce back to a simmer and cook, stirring often, until sauce is thickened, 10 to 15 minutes.
            \item Heat vegetable oil in a separate skillet over medium heat. 
                Stir chicken into the hot oil, sprinkle with curry powder, and sear chicken until lightly browned but still pink inside, about 3 minutes; stir often. 
                Transfer chicken and any pan juices into the sauce. 
                Simmer chicken in sauce until no longer pink, about 30 minutes; adjust salt and sugar to taste.
        }
		
	\recipe{Jambalaya}
	    \Ingredients{
			\item 2 tablespoons vegetable oil
			\item 1/2 pound andouille or other smoked sausage, cut into 1/2-inch slices
			\item 1/2 cup sliced celery
			\item 1 small onion, chopped
			\item 1 small red or green bell pepper, chopped
			\item 1 clove garlic, minced
			\item 1 3/4 cups chicken broth
			\item 1 cup diced fresh or canned tomatoes
			\item 1 bay leaf
			\item 1 teaspoon TABASCO® brand Original Red Sauce
			\item 1/4 teaspoon dried oregano leaves
			\item 1/4 teaspoon dried thyme leaves
			\item 1/8 teaspoon ground allspice
			\item 3/4 cup uncooked rice
			\item 1/2 pound shrimp, peeled, deveined and cut in half lengthwise
		}
		
		\Directions{
			\item Heat oil in a large heavy saucepan or Dutch oven over medium-high heat. 
				Add sausage, celery, onion, bell pepper, and garlic. 
				Cook 5 minutes or until vegetables are tender, stirring frequently.
			\item Stir in broth, tomatoes, bay leaf, TABASCO Sauce, oregano, thyme, and allspice. 
				Bring to a boil, reduce heat, and simmer uncovered for 10 minutes, stirring occasionally. 
			\item Stir in rice; cover and simmer 15 minutes. 
			\item Add shrimp; cover and simmer 5 minutes longer or until rice is tender and shrimp turn pink. 
			\item Let stand, covered, 10 minutes. 
				Remove bay leaf before serving.
		}
    
    \recipe{Pan-Roasted Soardfish Steaks}
        \Ingredients{
            \item 1/4 cup (1/2 stick) butter, room temperature
            \item 2 teaspoons chopped fresh parsley
            \item 1 garlic clove, minced
            \item 1/2 teaspoon ground mixed peppercorns, plus more for sprinkling
            \item 1/2 teaspoon (packed) grated lemon peel
            \item 1 tablespoon olive oil
            \item 4 1-inch-thick swordfish fillets (about 6 ounces each)
        }
        
        \Directions{
            \item Preheat oven to 400°F. 
            \item Mash butter, parsley, garlic, 1/2 teaspoon ground mixed peppercorns, and lemon peel in small bowl. 
                Season to taste with salt.
            \item Heat oil in heavy large ovenproof skillet over medium-high heat. 
                Sprinkle swordfish with salt and ground mixed peppercorns. 
            \item Add swordfish to skillet. 
                Cook until browned, about 3 minutes. 
            \item Turn swordfish over and transfer to oven. 
                Roast until just cooked through, about 10 minutes longer. 
                Transfer swordfish to plates. 
            \item Add seasoned butter to same skillet. 
                Cook over medium-high heat, scraping up browned bits, until melted and bubbling. 
                Pour butter sauce over swordfish and serve.
        }
    
    \recipe{Saag Paneer}
        \Ingredients{
            \item 1 teaspoon turmeric
            \item 1/2 teaspoon cayenne
            \item Kosher salt
            \item 3 tablespoons plus 1 1/2 tablespoons vegetable oil
            \item 12 ounces paneer, cut into 1-inch cubes
            \item 1 (16-ounce package) frozen chopped spinach
            \item 1 medium white onion, finely chopped
            \item 1 (1-inch thumb) ginger, peeled and minced (about 1 tablespoon)
            \item 4 cloves garlic, minced
            \item 1 large green serrano chile, finely chopped (seeds removed if you don't like it spicy!)
            \item 1/2 teaspoon store-bought or homemade garam masala, recipe follows
            \item 2 teaspoons ground coriander
            \item 1 teaspoon ground cumin
            \item 1/2 cup plain yogurt, stirred until smooth
        }
      
        \Directions{
            \item In a large bowl, whisk together the turmeric, cayenne, 1 teaspoon salt and 3 tablespoons oil. 
                Gently, drop in the cubes of paneer and gently toss, taking care not to break the cubes if you're using the homemade kind. 
                Let the cubes marinate while you get the rest of your ingredients together and prepped.
            \item Thaw the spinach in the microwave in a microwave-safe dish, 5 minutes on high, then puree in a food processor until smooth. 
                Alternatively, you can chop it up very finely with your knife.
            \item Place a large nonstick skillet over medium heat, and add the paneer as the pan warms. 
                In a couple of minutes give the pan a toss; each piece of paneer should be browned on one side. 
                Fry another minute or so, and then remove the paneer from the pan onto a plate.
            \item Add the remaining 1 1/2 tablespoons oil to the pan. 
                Add the onions, ginger, garlic and chile. 
                Now here's the important part: saute the mixture until it's evenly toffee-coloured, which should take about 15 minutes. 
                Don't skip this step - this is the foundation of the dish! 
                If you feel like the mixture is drying out and burning, add a couple of tablespoons of water.
            \item Add the garam masala, coriander and cumin. 
                If you haven't already, sprinkle a little water to keep the spices from burning. 
                Cook, stirring often, until the raw scent of the spices cook out, and it all smells a bit more melodious, 3 to 5 minutes.
            \item Add the spinach and stir well, incorporating the spiced onion mixture into the spinach. 
                Add a little salt and 1/2 cup of water, stir, and cook about 5 minutes with the lid off.
            \item Turn the heat off. 
                Add the yogurt, a little at a time to keep it from curdling. 
                Once the yogurt is well mixed into the spinach, add the paneer. 
                Turn the heat back on, cover and cook until everything is warmed through, about 5 minutes. Serve.
        }
        
    \recipe{Thai Red Curry}
        \Ingredients{
            \item 1 1/4 cups brown jasmine rice or long-grain brown rice, rinsed
            \item 1 tablespoon coconut oil or olive oil
            \item 1 small white onion, diced
            \item Pinch of salt, more to taste
            \item 1 tablespoon finely grated fresh ginger (about a 1-inch nub of ginger)
            \item 2 cloves garlic, pressed or minced
            \item 1 red bell pepper, sliced into thin 2-inch long strips
            \item 1 yellow, orange or green bell pepper, sliced into thin 2-inch long strips
            \item 3 carrots, peeled and sliced on the diagonal into 1/4-inch thick rounds (to yield about 1 cup sliced carrots)
            \item 2 tablespoons Thai red curry paste
            \item 1 can (14 ounces) regular coconut milk
            \item 1/2 cup water
            \item 1 1/2 cups packed thinly sliced kale (tough ribs removed first), preferably the Tuscan/lacinato/dinosaur variety
            \item 1 1/2 teaspoons coconut sugar or turbinado (raw) sugar or brown sugar
            \item 1 tablespoon tamari or soy sauce
            \item 2 teaspoons rice vinegar or fresh lime juice
            \item Garnishes/sides: handful of chopped fresh basil or cilantro, optional red pepper flakes, optional sriracha or chili garlic sauce
        }
        
        \Directions{
            \item To cook the rice, bring a large pot of water to boil. 
                Add the rinsed rice and continue boiling for 30 minutes, reducing heat as necessary to prevent overflow. 
                Remove from heat, drain the rice and return the rice to pot. 
                Cover and let the rice rest for 10 minutes or longer, until you're ready to serve. 
                Just before serving, season the rice to taste with salt and fluff it with a fork.
            \item To make the curry, warm a large skillet with deep sides over medium heat. 
                Once it’s hot, add the oil. 
                Add the onion and a sprinkle of salt and cook, stirring often, until the onion has softened and is turning translucent, about 5 minutes. 
                Add the ginger and garlic and cook until fragrant, about 30 seconds, while stirring continuously.
            \item Add the bell peppers and carrots. 
                Cook until the bell peppers are fork-tender, 3 to 5 more minutes, stirring occasionally. 
                Then add the curry paste and cook, stirring often, for 2 minutes.
            \item Add the coconut milk, water, kale and sugar, and stir to combine. 
                Bring the mixture to a simmer over medium heat. 
                Reduce heat as necessary to maintain a gentle simmer and cook until the peppers, carrots and kale have softened to your liking, about 5 to 10 minutes, stirring occasionally.
            \item Remove the pot from the heat and season with tamari and rice vinegar. 
                Add salt (I added 1/4 teaspoon for optimal flavor), to taste. 
                If the curry needs a little more punch, add ½ teaspoon more tamari, or for more acidity, add 1/2 teaspoon more rice vinegar. 
                Divide rice and curry into bowls and garnish with chopped cilantro and a sprinkle of red pepper flakes, if you'd like. 
                If you love spicy curries, serve with sriracha or chili garlic sauce on the side
        }
    
    \recipe{Whole Roasted Red Snapper}
        \Ingredients{
            \item 4-pound whole red snapper, head on, scales removed and cleaned
            \item 1/3 cup canola oil
            \item 1 clove garlic, sliced
            \item 1 tablespoon grated ginger
            \item 1 tablespoon ground coriander seeds
            \item 1/2 teaspoon chili flakes
            \item 3 tablespoons coconut milk
            \item juice and zest of 1 lime
            \item 1/4 cup basil chiffonnade
            \item 1/4 cup cilantro chiffonnade
            \item fine sea salt and freshly ground black pepper 1 lime
        }
        
        \Directions{
            \item Preheat oven to 400°F.
            \item Generously season inside the belly and both sides of the fish with salt and pepper. 
                Place the fish on a roasting pan.
            \item Combine the canola oil with the garlic, ginger, coriander, chili flakes, coconut milk, lime juice and zest in a mixing bowl and stir to combine.
            \item Spoon the spice mixture over the snapper and bake for about 25-30 minutes, basting frequently. 
                Bake the fish until a metal skewer can easily be inserted into the fish and, when left in for 5 seconds, feels warm.
            \item While the fish is roasting, cook the rice. 
                First, place the rice in a fine sieve and rinse under cool water until the water begins to run clear. 
                Transfer rice to a medium pot; add water, season with a pinch of salt, add lemongrass and bring to a boil over medium-high heat. 
                Lower the heat and simmer for 10 minutes. Remove the rice from heat, cover and let sit for another 10 minutes. 
                When the rice is cooked, remove and discard lemongrass. 
                Add the coconut milk and cilantro, gently stir to incorporate and season to taste with lime juice, salt and pepper.
            \item To serve, spoon the coconut rice on to the middle of four plates. 
                Fillet the snapper, running a knife lengthwise down the fish at about the center, to separate the side into two fillets, and then under the flesh to separate it from the bone. 
                Carefully lift off each fillet and place on top of the rice. 
                When the top fillets have been removed, lift off the fish bones and portion the bottom in the same manner. 
                Plate the remaining fillets and spoon some of the sauce from the roasting pan over each portion. 
                Finish each dish with a squeeze of fresh lime juice and garnish with chiffonade of basil and cilantro.
        }
        
\chapter{Dessert}
    \recipe{Chocolate Lava Cake}
        \Ingredients{
            \item 6 (1-ounce) squares bittersweet chocolate
            \item 2 (1-ounce) squares semisweet chocolate
            \item 10 tablespoons (1 1/4 stick) butter
            \item 1/2 cup all-purpose flour
            \item 1 1/2 cups confectioners' sugar
            \item 3 large eggs
            \item 3 egg yolks
            \item 1 teaspoon vanilla extract
            \item 2 tablespoons orange liqueur
        }
    
        \Directions{
            \item Preheat oven to 425 degrees F. Grease 6 (6-ounce) custard cups. 
            \item Melt the chocolates and butter in the microwave, or in a double boiler. 
                Add the flour and sugar to chocolate mixture. Stir in the eggs and yolks until smooth. 
                Stir in the vanilla and orange liqueur. 
            \item Divide the batter evenly among the custard cups. 
                Place in the oven and bake for 14 minutes. 
                The edges should be firm but the center will be runny. 
                Run a knife around the edges to loosen and invert onto dessert plates.
        }
  
    \recipe{Tiramisu}
        \Ingredients{
            \item 6 egg yolks 
            \item 3/4 cup white sugar 
            \item 2/3 cup milk 
            \item 1 1/4 cups heavy cream 
            \item 1/2 teaspoon vanilla extract 
            \item 1 pound mascarpone cheese 
            \item 1/4 cup strong brewed coffee, room temperature
            \item 2 tablespoons rum 
            \item 2 (3 ounce) packages ladyfinger cookies 
            \item 1 tablespoon unsweetened cocoa powder
        }
        
        \Directions{
            \item In a medium saucepan, whisk together egg yolks and sugar until well blended. 
                Whisk in milk and cook over medium heat, stirring constantly, until mixture boils. 
                Boil gently for 1 minute, remove from heat and allow to cool slightly. 
                Cover tightly and chill in refrigerator 1 hour.
            \item In a medium bowl, beat cream with vanilla until stiff peaks form. 
                Whisk mascarpone into yolk mixture until smooth.
            \item In a small bowl, combine coffee and rum. 
                Split ladyfingers in half lengthwise and drizzle with coffee mixture.
            \item Arrange half of soaked ladyfingers in bottom of a 7x11 inch dish. 
                Spread half of mascarpone mixture over ladyfingers, then half of whipped cream over that. 
                Repeat layers and sprinkle with cocoa. Cover and refrigerate 4 to 6 hours, until set.
        }

\end{document}